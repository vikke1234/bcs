\documentclass{paper}
\usepackage{hyperref}
\usepackage{graphicx}
\usepackage[
backend=biber,
style=numeric,
sorting=ynt
]{biblatex}

\graphicspath{{./images/}}
\addbibresource{refs.bib}

\title{TBD}
\author{Viktor Horsmanheimo}
\date{\today}

\begin{document}
\maketitle
\tableofcontents
\newpage

\section{Introduction}
Field programmable gate arrays(FPGA) have become more popular in the past few
decades. FPGAs essentially are chips that unlike standard ones for example
CPUs, are programmable after manufacturing. FPGAs consist of logic gate arrays
which perform simple boolean algebra. Between these gates we have interconnect
which allows the data to be moved between the logic gates. We will go into
details about the internals of these components in the next section.

FPGAs were created in the 80s by Altera, at the time they weren't very
practical because the size of the transistors were very large. This meant that
you couldn't fit very much on a reasonably sized chip. But as we know with
Moore's law the transistor size halves every two years, this has affected FPGAs
as well and now contains enough transistors to compete with application
specific integrated circuits (ASIC). ASICs are in general faster than FPGAs if
you are working with the latest technology. Due to the fact that it is risky to
work with the newest technology, manufacturers rarely use it. This is where
FPGAs are useful, since they're reprogrammable they can always use the latest
technology. If something goes wrong you can always change it. This is one of
the biggest reasons FPGAs have become increasingly popular.

\section{Outline}

\section{FPGA}
\subsection{Components}
In the introduction we talked about FPGAs consisting of
logic gates, we are now going to go into more detail about how they work. As we
know any boolean algebra function can be expressed as a lookup
table\cite{m_d_mano_digital_2012}. Because of this property FPGAs extensively
use lookup tables(LUT). The way LUTs are implemented are using multiplexers,
they map a number of inputs to a single output. There's been a lot of testing
as to how many inputs a LUT should take.

\printbibliography

\end{document}
