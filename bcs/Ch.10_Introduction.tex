\chapter{Introduction\label{intro}}

Field programmable gate arrays (FPGA) are chips that unlike standard ones such as
CPUs, are programmable after manufacturing. Because of this they have
become increasingly more popular in the past few decades. FPGAs consist of
logic gate arrays which in theory can evaluate any boolean function. Between
these gates we have wires and control blocks which allow the data to be moved
between the logic gates. We will present the internals of these components in
the next section.

FPGAs were created in the 80s by Altera, at that time they weren't very
practical due to the size of the transistors being very large. This meant that
a chip couldn't fit many transistors and were limited in what they could
do. Moore's law states that the transistor size halves every two years, this
has affected FPGAs as well and they now contain enough transistors to compete
with application specific integrated circuits (ASIC). An ASIC is a type of
hardware that is made specifically for a task, for example cryptographic
currency mining hardware. These chips are in general faster and use less energy
than FPGAs if the manufacturer is working with the latest technology. But
because ASICs cost a lot of money to create and if something goes wrong with
the chip it cannot be programmatically fixed. It is extremely expensive to
rectify such a mistake. This is where FPGAs are very useful as they're
reprogrammable, ASICs can be first implemented on an FPGA and once the chip has
been verified that it works, it can be ported to an ASIC.

\section{Outline\label{outline}}
In section \ref{FPGA} we will go into detail about how FPGAs work
internally, how they're configured and the technologies behind it. In
\ref{crypto} we will give an overview about crypto and how reconfigurable
hardware can be used with it.
