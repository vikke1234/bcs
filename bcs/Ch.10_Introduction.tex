\chapter{Introduction\label{intro}}

\textit{Field Programmable Gate Arrays} (FPGA) are chips that unlike standard
ones such as CPUs, are programmable after manufacturing. Because of property,
FPGAs have become increasingly more popular in the past few decades. FPGAs
consist of logic gate arrays which in theory can evaluate any boolean function.
In this thesis, we will give an overview on FPGAs with a focus on post-quantum
cryptography.

FPGAs were developed in the 80s by Altera, at that time they were not very
practical due to the size of the transistors being very large. This means that
a chip could not fit many transistors and were limited in what they could do.
Moore's law states that the transistor size halves every two years, this has
affected FPGAs as well and they now contain enough transistors to compete with
\textit{Application Specific Integrated Circuits} (ASIC). An ASIC is a type of
hardware that is made specifically for a given task, for example,
cryptocurrency mining hardware. These chips are in general faster and use less
energy than FPGAs if the manufacturer is working with the latest technology.
However design mistakes in ASICs are costly and it is extremely expensive to
rectify mistakes that end up in produced hardware. This is where FPGAs show
their versatility as they are reprogrammable and can be used to detect and
correct design errors. Once the chip has been verified that it works, it can be
implemented as an ASIC. This is an effective method of producing bug-free
ASICs.

With the recent development of quantum computers, the problems we considered
hard are no longer hard. This means we will need new problems, examples of
these problems are error correcting codes, lattice based cryptography and
multivariate polynomials.

In Chapter \ref{FPGA} we will go into detail about how FPGAs work internally,
how they are configured, and the technologies behind them. In Chapter
\ref{crypto} we will give a brief overview on cryptography, the issues quantum
computers bring, and solutions for these problems. Finally in Chapter
\ref{FPGA_crypto} we will show an example of a fast implementation of a
post-quantum cryptosystems.
