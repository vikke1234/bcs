\chapter{Introduction\label{intro}}

\textit{Field Programmable Gate Arrays} (FPGA) are chips that unlike standard
ones such as CPUs, are programmable after manufacturing. Because of this they
have become increasingly more popular in the past few decades. FPGAs consist of
logic gate arrays which in theory can evaluate any boolean function. In this
thesis we will give an overview on FPGAs with a focus on cryptography.

FPGAs were developed in the 80s by Altera, at that time they were not very
practical due to the size of the transistors being very large. This meant that
a chip could not fit many transistors and were limited in what they could do.
Moore's law states that the transistor size halves every two years, this has
affected FPGAs as well and they now contain enough transistors to compete with
\textit{Application Specific Integrated Circuits} (ASIC). An ASIC is a type of
hardware that is made specifically for a task, for example cryptographic
currency mining hardware. These chips are in general faster and use less energy
than FPGAs if the manufacturer is working with the latest technology. ASICs are
costly, if anything goes wrong after production it is extremely expensive to
rectify such a mistake. This is where FPGAs are very useful as they are
reprogrammable, ASICs can be first implemented on an FPGA. Once the chip has
been verified that it works, it can be ported to an ASIC. This minimizes the
amount of bugs found after the ASIC is manufactured.

\section{Outline\label{outline}}
In Chapter \ref{FPGA} we will go into detail about how FPGAs work internally,
how they are configured and the technologies behind it. In Chapter \ref{crypto}
we will give an overview about symmetric, asymmetric cryptography and how
reconfigurable hardware can be used with it.
