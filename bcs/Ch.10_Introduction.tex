\chapter{Introduction\label{intro}}

Field programmable gate arrays (FPGA) have become increasingly more popular in
the past few decades. FPGAs are chips that unlike standard ones for example
CPUs, are programmable after manufacturing. FPGAs consist of logic gate arrays
which evaluate simple boolean function. Between these gates we have wires and
control blocks which allows the data to be moved between the logic gates. We
will present the internals of these components in the next section.

FPGAs were created in the 80s by Altera, at that time they weren't very
practical because the size of the transistors were very large. This meant that
you couldn't fit very much on a reasonably sized chip. Moore's law states that
the transistor size halves every two years, this has affected FPGAs as well and
now they contain enough transistors to compete with application specific
integrated circuits (ASIC). ASICs are hardware that is made specifically for a
task, for example cryptographic currency mining hardware. These chips are in
general faster than FPGAs if you are working with the latest technology. Due to
the fact that it is risky to work with the newest technology, manufacturers
rarely use it. This is where FPGAs are useful, since they're reprogrammable
they can always use the latest technology. If something goes wrong you can
always change it. Because of this FPGAs continue to rise in popularity. % TODO add picture of ASIC vs FPGA nano meters

\section{Outline\label{outline}}
In section \ref{FPGA} we will go into detail about how FPGAs work
internally, how they're configured and the technologies behind it.
