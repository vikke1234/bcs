\chapter{Conclusion\label{conclusion}}
In this thesis we present an efficient hardware implementation of the LWE
cryptosystem. The design choices made for the implementation reduces the
latency and hardware cost by a large margin. The polynomial multiplier played a
large role in the optimization. The implementations are roughly $20\% - 78\%$
faster than the previous implementations \citep{FPGA_Post_Quantum_Primitives}.

We can see that FPGAs are very useful as they provide the means to quickly
compute matrix multiplication due to being very good at parallelization. FPGAs
are also very useful currently as we do not know whether an algorithm is
breakable by quantum computers, so creating ASICs of it is risky. Changing FPGA
functionality is trivial, while recreating an ASIC implementation. Future work
would consist of implementing the NIST competition winner.
