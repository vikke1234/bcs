\chapter{Conclusion\label{conclusion}}

$q = 12289$ is considered a safe prime number to use. Although any $q > 10 000$
with each polynomial having $128 - 1024$ coefficients gets you the security
level of at least 112 bits as recommended by NIST
\citep{FPGA_Post_Quantum_Primitives}. This cryptosystem was implemented on a
Zynq-7000 FPGA, we can see the LUTs needed for different sizes of $n$ in Table
\ref{tab:hardwarecost}.

\begin{table}[H]
    \centering
    \begin{tabular}{l|ll}
        \textit{n} & LUTs   & Registers \\ \hline
        128        & 66251  & 16805     \\
        256        & 11490  & 33138     \\
        512        & 227458 & 65643     \\
        1024       & 426402 & 130540
    \end{tabular}
    \caption{Hardware cost using only LUTs where $q = 12289$
    \citep{FPGA_Post_Quantum_Primitives}}
    \label{tab:hardwarecost}
\end{table}

The presented thesis attempts to lower the barrier of entry to post quantum
cryptography using FPGAs. We also present efficient hardware modules that are
roughly $20\% - 78\%$ faster than the previous implementations
\citep{FPGA_Post_Quantum_Primitives}
