\chapter{Conclusion\label{conclusion}}

This thesis covers the basics of FPGAs as well as a brief introduction of the public key 
crypto system. We

We can see that FPGAs are very useful as they
provide the means to quickly compute matrix multiplication due to being very
good at parallelization. FPGAs are also very useful currently as we do not know whether an algorithm
is breakable by quantum computers, so creating ASICs of it is risky. Changing FPGA functionality
is trivial, while recreating an ASIC implementation
We also present efficient hardware modules that
are roughly $20\% - 78\%$ faster than the previous implementations
\citep{FPGA_Post_Quantum_Primitives}.
