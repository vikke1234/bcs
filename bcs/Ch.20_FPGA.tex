\chapter{FPGA\label{FPGA}}
As briefly mentioned in the introduction, FPGAs consist of an array of logic
gates. In this section, we will give an overview of how such components work
and different ways of configuring them. Modern FPGAs may also include other
components, such as processors and multipliers which is out of the scope of the
present thesis.

\begin{figure}[H]
    \centering
    \includegraphics[scale=0.5]{fpga_structure.png}
    \caption{An example of an island-style FPGA\citep{M.MorrisMano3}}
    \label{fig:fpga_structure}
\end{figure}


\section{Lookup tables}

\textit{Lookup tables} (LUT), are a crucial component of FPGAs. To fully
understand LUTs, we will first introduce some preliminaries on \textit{boolean
functions}.

\begin{definition}
    A boolean function is a function that is defined on two element sets, i.e.
    $\{0,\ 1\}$, $\{\text{true},\ \text{false}\}$, ..., etc.

    Without loss of generality, we can define a boolean function $f$ to be a
    function of the form

    \[f \colon \{0,\ 1\}^N \rightarrow \{0,\ 1\}, \quad \text{where $N$ is the
    size of the input}\]

    Any boolean function can be represented by a \textit{truth table}. A truth
    table represents all the possible outputs when a function $f$ is evaluated.
    Consequently, the size of the truth table is $2^N$.
\end{definition}

Hereon the addition $+$ and multiplication $\cdot$ is understood as the
usual boolean addition and multiplication.

\begin{example}
    \label{ex:boolean_function}
    We consider the following function

    \begin{equation} \label{eq:examle_eq}
    f \colon \{0,\ 1\}^3 \rightarrow \{0,\ 1\},
        \quad (A, B, C) \mapsto A + B \cdot C.
    \end{equation}

    There are $2^3 = 8$ possible values for the function $f$. Table
    \ref{tab:example_truth_table} represents the truth table of $f$.

    \begin{table}[H]
        \centering
        \begin{tabular}{|l|l|l|l|}
            \hline
            A & B & C & Out \\ \hline
            0 & 0 & 0 & 0   \\ \hline
            0 & 0 & 1 & 1   \\ \hline
            0 & 1 & 0 & 0   \\ \hline
            0 & 1 & 1 & 1   \\ \hline
            1 & 0 & 0 & 0   \\ \hline
            1 & 0 & 1 & 1   \\ \hline
            1 & 1 & 0 & 1   \\ \hline
            1 & 1 & 1 & 1   \\ \hline
        \end{tabular}
        \caption{Truth table for function $f$}
        \label{tab:example_truth_table}
    \end{table}

    In fact, these truth tables are what allow the reconfiguration of
    FPGAs, by replacing the values in truth tables we can make instruct the
    FPGA to evaluate any boolean function. We'll look at how this could be
    represented using logic gates in a later section.

\end{example}

\begin{definition}

    A \textit{lookup table} is a truth table that is represented in continuous
    memory. LUTs are implemented are using $N:1$ \textit{multiplexers}. A
    multiplexer is a component that simply maps multiple inputs into a single
    output, the internals of it are out of the scope of this thesis. Using
    $N$-bits of memory to store all the answers of the truth table in memory,
    we can evaluate any boolean function.

\end{definition}

To implement Equation \ref{eq:examle_eq} for an \textit{Application Specific
Integrated Circuit} (ASIC), we could do the following; in Figure
\ref{fig:logic_gate_explanation} we can see how logic gates behave when
different signals are sent through them. When we implement boolean functions,
addition is an OR gate, multiplication is an AND gate, and negation is a NOT
gate. So if we go back to Example \ref{ex:boolean_function} we can represent it
using an OR and an AND gate as seen in Figure \ref{fig:logic_gate_impl}. We
first calculate $B \cdot C$ after which we send the result to the or gate and
calculate $A + B \cdot C$. ASICs then use this to implement all boolean
functions.

\begin{figure}[H]
    \centering
    \begin{subfigure}[b]{.8\textwidth}
        \centering
        \includegraphics[scale=0.20]{logic_gate_explanation.jpg}
        \caption{An explanation of how different logic gates behave
            \citep{LogicGateBehavior}}
        \label{fig:logic_gate_explanation}
    \end{subfigure}
    ~
    \begin{subfigure}[b]{.4\textwidth}
        \centering
        \includegraphics[scale=0.6]{circuit.png}
        \caption{Logic gate implementation of the function in Example
            \ref{ex:boolean_function}}
        \label{fig:logic_gate_impl}
    \end{subfigure}
\end{figure}


\subsection{Delay flip-flop}
LUTs alone are not enough to build an efficient FPGA as we cannot store
state. That is where the \textit{delay(D) flip-flops} come in. A D flip-flop
can store a single bit of data, this is needed as it can be used to
store state. In Figure \ref{fig:lut_flipflop} we give an example of a D
flip-flop. The input for the flip-flop is D, the output is Q, and the triangle
underneath D is the clock. The D flip-flop works around the clock, it can only
read the input whenever the clock is on a \textit{rising edge}. A rising edge
is when the clock sends a signal to the flip-flop to tell it to check the data
value. We illustrate the above in Figure \ref{fig:flipflop_func}. The output
Q is only changed to "high" after the second cycle, and waits a cycle to go to
"low" after the third cycle.

\begin{figure}[H]
    \begin{subfigure}[b]{.4\textwidth}
        \centering
        \includegraphics[scale=0.4]{lut_structure.png}
        \caption{An example of how a lookup table could be implemented}
        \label{fig:lut_flipflop}
    \end{subfigure}
    ~
    \centering
    \begin{subfigure}[b]{.4\textwidth}
        \centering
        \includegraphics[scale=0.13]{flipflop_functionality.png}
        \caption{How a flip-flop reads the output}
        \label{fig:flipflop_func}
    \end{subfigure}
\end{figure}


Next we will present some other parts of FPGAs.

\begin{enumerate}
    \item

        \textit{Input/Output} (I/O blocks, IOB) blocks are blocks that are placed
        around the periphery of the FPGA. They control the interface between
        the FPGA and external circuits such as the clock and power usage.

    \item

        \textit{Connection blocks} (CB) are placed along the wires by the logic
        blocks. These blocks control which data goes in and out of a logic
        block. They also connect the I/O blocks to the FPGA. Figure
        \ref{fig:fpga_structure} shows an example of connection block
        placement.

    \item

        \textit{Switch block} (switch box, SB) is a type of block that is
        placed at every intersection in the wiring. As the name suggests it is
        built from a bunch of switches and its job is to route the electricity
        between the logic blocks. There are different types of implementations
        for switch blocks, depending on the type it will have different levels
        of connectivity and efficiency. Though the specifics of this are out of
        the scope of this thesis.

\end{enumerate}

There are multiple ways the logic blocks can be connected, for example,
Xilinx FPGAs have four different types of interconnects: long lines,
hex lines, double lines, and direct lines. The direct lines are to
connect neighboring logic gates for fast transfer. Hex and double lines
connect logic gates that are a medium length away. Long lines go along
the entire chip and are often used for global signals
\citep{HideharuAmano8}.

\section{Configuration}

We will now give a brief overview on how FPGAs can be configured, we will
mainly to focus: static RAM, flash memory, and antifuse. Depending on the
technology used they can be more or less configurable. Every configurable
element in FPGAs require 1 bit to maintain the configuration. For example LUTs
need to be filled, switch blocks need to be enabled or disabled etc. These are
configured with a "binary file", which is essentially just a \textit{bitstream}
that set the different elements correctly in the FPGA. The exact format for the
bitstream is vendor specific and may not be known
\citep{ScottHauckAndreDeHon5}.

\subsection{Static RAM}

\textit{Static RAM} (SRAM) is a very common way to handle configuration,
simply write the bitstream into SRAM cells. There are some issues with SRAM, it
is volatile which means that if power is lost so is the configuration. Another
issue is power consumption, as the SRAM cell contains 6-12 transistors compared
to flash memory which only requires roughly two transistors
\citep{HideharuAmano8}.

\subsection{Flash memory}

Flash memory is a way to store the configuration in memory without needing
power for the chip to remember it. It works by having electrically erasable
programmable read-only memory, its benefit is that the configuration can be
remembered even if the device is powered off. The downside is that there is
only a limited amount of writes before it fails and that flash is often slower
at writing than SRAM \citep{M.MorrisMano3}.

\subsection{Antifuse}

An antifuse is the opposite of a fuse, it works by having high resistance and
not allowing electricity to pass until the fuse is blown. This technology is
very reliable and rarely corrupts, so it is often used in places with radiation
\citep[Ch. 1]{ScottHauckAndreDeHon5}. The downside of antifuses is that they
can only be programmed once, after it's programmed the fuses are blown and can
not be restored.
