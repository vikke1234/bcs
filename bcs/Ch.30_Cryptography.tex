% What is symmetric and assymetric cryptography
\chapter{Cryptography\label{crypto}}
Cryptography has always been important, it is been used since ancient times to
secure communication. Cryptography can be used for a number of things like
verifying data integrity i.e making sure that the data is not sent by an outside
party etc. Especially in today's world where vast amounts of information is
constantly transferred where tampering with information can have catastrophic
consequences. Thankfully cryptography gives us the tools to safely distribute
and receive information. In this chapter we will discuss different forms of
cryptography . FPGAs can make for example hash functions among other things
faster. With quantum computers in development the current cryptographic schemes
will be obsolete as, for example RSA can be broken in polynomial time.

One of the earliest forms of cryptography was the Ceasar cipher, invented by
Julius Ceasar in ancient Rome. The Ceasar cipher is a \textit{symmetric
cipher}. A symmetric cipher is a cipher where users agree on a key that is used
to encrypt and decrypt messages. The cipher is based around shifting letters by
$N$ letters in the alphabet. The Romans only had to agree on how many letters
they should shift. A modern example of a secret key encryption scheme is the
\textit{Advanced Encryption Standard} (AES). The issue in today's world is
exchanging keys in order to keep the decryption keys secret. As means for
keeping exchanging keys \textit{public key cryptography} has become an
extremely powerful tool \citep{FranciscoRodriguez-Henriquez10}.

When we encrypt messages, there are two different types of ciphers we can use,
stream- and block ciphers. A stream cipher is a type of cipher that encrypts a
bit or in some scenarios a byte at a time. Block ciphers on the other hand take
a block of data and transform into an unreadable block of the same length.


\section{Public key cryptography}
Cryptography was revolutionized in 1976 when the \textit{Diffie and Hellman key
exchange} was invented. It solved the issues of exchanging keys as well as
providing the means to implement authentication. \textit{Public key
cryptography}, also know as \textit{asymmetric cryptography} is a type of
cryptography that works by having two sets of keys, a public- and a private key
that are used for encryption and decryption. As the name implies, the public
key is known to everyone while the private key should not be shared.

\begin{definition}

    The Diffie and Hellman algorithm works as follows. Let $g$ be a generator
    element for the group $G$ of integers modulo $p$. Let Alice pick a random
    natural number $a$ and sends $g^a$ to Bob. Bob then picks a random number
    $b$ and sends $g^b$ to Alice. Alice then computes $g^{ab}$ and Bob computes
    $g^{ba}$ \citep{FranciscoRodriguez-Henriquez10}.

\end{definition}

When we encrypt something we assume some function is hard for a computer to
reverse. Due to the recent developments in quantum computing some of the
problems we considered hard, such as integer factorization as well as the
Diffie and Hellman key exchange can be broken in polynomial time
\citep{FPGA_Post_Quantum_Primitives}.
This means we have to quickly move away from the methods we currently use.
There are a few cryptographic functions that have gotten popular that are hard
even for quantum computers. An example of a hard problem for quantum computers
is a lattice based problem, \textit{Ring Learning With Errors} (R-LWE)
\citep{FPGA_Post_Quantum_Primitives}.

\subsection{Hash functions}
\begin{definition}

A \textit{hash function} H is a one-way function that takes an arbitrary length
bit string B, a \textit{message}. Then it creates a unique bit string of fixed
length, H(B) is a hash or digest of B \cite{FranciscoRodriguez-Henriquez10}.

\end{definition}

I.e. hash functions create a hash, a unique string of bytes from a given input.
Hash functions are not perfect, although having a very small chance of two
inputs having the same output it can happen. This is a terrible thing for hash
functions as a malicious user could then fake for example a private key.

