% What is symmetric and assymetric cryptography
\chapter{Cryptography\label{crypto}}
Cryptography has always been important, it is been used since ancient times to
secure communication. Cryptography can be used for a number of things like
verifying data integrity i.e making sure that the data is not sent by an outside
party etc. Especially in today's world where vast amounts of information is
constantly transferred where tampering with information can have catastrophic
consequences. Thankfully cryptography gives us the tools to safely distribute
and receive information. In this chapter we will discuss different forms of
cryptography . FPGAs can make for example hash functions among other things
faster. With quantum computers in development the current cryptographic schemes
will be obsolete as, for example RSA can be broken in polynomial time.

One of the earliest forms of cryptography was the Ceasar cipher, invented by
Julius Ceasar in ancient Rome. The Ceasar cipher is a \textit{secret key
cryptography} where users agree on a key that is used to encrypt and decrypt
messages. The cipher is based around shifting letters by $N$ letters in the
alphabet. The Romans only had to agree on how many letters they should shift. A
modern example of a secret key encryption scheme is \textit{Data Encryption
Algorithm} (DEA) which used 56 bit keys which is way too small and is being
deprecated. The issue in today's world is exchanging keys in order to keep the
decryption keys secret. As means for keeping single decryption keys secret have
minimized \textit{public key cryptography} has become extremely useful
\citep{FranciscoRodriguez-Henriquez10}.

There are two types of ciphers, stream- and block ciphers. A stream cipher is
a type of cipher that encrypts a bit or in some scenarios a byte at a time.
While block ciphers take a block of data and transform into an unreadable block
of the same length.

\section{Public key cryptography}
Cryptography was revolutionized in 1976 when the Diffie and Hellman key
exchange was invented. It solved the issues of exchanging keys as well as
providing the means to implement authentication. Public key cryptography is a
type of cryptography that works by having two sets of keys, a public- and a
private key that are used for encryption and decryption. When we encrypt
something we assume some operation is hard for a computer to reverse. Due to
the recent developments in quantum computing some of the problems we considered
hard, such as integer factorization are no longer hard. There are new problems
that are hard even for quantum computers, such as the \textit{Ring Learning
With Errors} (R-LWE) \citep{FPGA_Post_Quantum_Primitives}.

\begin{definition}

    The Diffie and Hellman algorithm works as follows. Let $g$ be a generator
    element for the group $G$ of integers module $p$. Let Alice pick a random
    natural number $a$ and sends $g^a$ to Bob. Bob then picks a random number
    $b$ and sends $g^b$ to Alice. Alice then computes $g^{ab}$ and Bob computes
    $g^{ba}$ \citep{FranciscoRodriguez-Henriquez10}.

\end{definition}

\subsection{Asymmetric algorithms}
\textit{Asymmetric algorithms} are algorithms that use the public and private
key, where the private key isn't possible to derive from the public one. The
public key is known to everyone when sending a message,
