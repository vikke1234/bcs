\documentclass{paper}
\usepackage[T1]{fontenc}

\title{Postkvantkryptografi}
\author{Viktor Horsmanheimo}
\date{\today}

\begin{document}
\maketitle

Kryptografi har alltid varit en viktig del i olika delar av samhället.Genom att
kryptera tex text eller data, blir de oförståeliga till de dekrypteras med en
nyckel. Kryptografi säkerställer att inte någon utomstående kan avlyssna och
direkt tolka hemlig konversation eller data. En av de tidigaste formerna av
kryptografi man upptäckt förekom i det forna romarriket. Julius Caesar använde
sig av den så kallade Caesarchiffern. Chiffern fungerade genom att man flyttade
på bokstäverna till exempel tre steg framåt.Då bir A, D och B blir E och så
vidare. Detta är dock inte väldigt säkert i moderna tider, då nyckel- rymden
bara är längden av alfabetet. Det var dock tillräckligt säkert på den tiden.
Caesarchiffern är ett substitionschiffer. Det är en typ av chiffer där man
substituerar bokstäver med andra bokstäver. Ett modernare exempel på ett
substitionschiffer är Enigma som skapades under II världskriget i Tyskland. Vi
kan se hur stor inverkan det hade på andra världs kriget då Alan Turing
lyckades bryta Enigman och England kunde avlyssna all tysk kommunikation.

År 1994 uppfann Peter Shor en algoritm, som i teorin kunde användas för att
bryta våra nuvarande kryptografiska system om man hade en tillräckligt bra
kvantdator. Även om vi inte ännu har en kvantdator som kan använda sig av
algoritmen, är det bara en tidsfråga innan någon organisation skapar en
kvantdator som klarar av det. Det finns även problemet med att avlyssna
krypterad kommunikation nu. Avlyssnaren, som vill komma åt krypterad data, är
tvungen att vänta tills en tillräckligt snabb kvantdator, som klarar av att
bryta krypteringen utvecklas och avkrypterar informationen.

Detta betyder att i princip all vår nuvarande kryptografi kommer att vara
brytbar inom några decennier. Vi behöver alltså nya problem som tros vara svåra
även kvantdatorer. På grund av att vi inte har något sätt att verifiera att ett
problem är svårt för en kvantdator så kan vi dock inte vara alldeles säkra på
att problemet inte är lösbart.

\textit{National Institute of Standards and Technology} (NIST) ordnar en
tävling där man försöker hitta nya kryptografiska system som är resistenta för
kvantdatorer. Tävlingen är nu i tredje och sista omgången. Fyra av deltagarna
använder sig av gitterbaserade lösningar. En, Classic McElice använder sig av
felkorrigerande koder. Man lägger till fel, som man sedan med hjälp av rätt
nyckel, enkelt kan ta bort. Vi kommer dock inte att fokusera desto mera på
felkorrigerande koder.

Ett gitter är den oändliga mängd punkter som man kan generera av en mängd
linjärt oberoende vektorer. Det ser i princip ut som ett rutnät. Gitterbaserade
problem är till exempel \textit{Lärande Med Fel}. Oded Regev bevisade år 2005
att lärande med fel var ett svårt problem för kvantdatorer. Han bevisade att om
man skulle kunna lösa lärande med fel snabbt, så skulle man även kunna lösa
\textit{kortaste vektor problemet} snabbt. Kortaste vektor problemet är ett
problem där man söker kortaste vektorn inom gittret. Om man ser på ett
2-dimensionellt gitter är det väldigt enkelt att se den kortaste vektorn. Om
man har tusentals dimensioner blir det dock väldigt svårt att räkna ut den
kortaste vektorn, även för en dator.

Lärande med fel fungerar på följande sätt: Alice väljer en allmän vektor $a$ av
storlek $n$, en hemlig vektor $s$ av storlek $n$ och ett fel $e$. Alice räknar
sedan ut $b = a \cdot s + e$. Alice publicerar sedan $a$ och $b$. Det Regev
bevisade är att givet $a$ och $b$ så är det svårt att hitta $s$. Detta problem
använder sig mycket av matrismultiplikation. Komplexiteten av
matrismultiplikation är $\mathcal{O}(n^2)$. Genom att se på matriserna som
polynom, kan vi minska komplexiteten till $\mathcal{O}(n \log n)$ med hjälp av
\textit{Diskreta Fourier Transformen} (DFT). Kryptografiska systemen SABER och
KYBER är finalister i NIST tävlingen som använder sig av lärande med fel.

Det är estimerat att det tar cirka 10 till 20 år att förflytta sig till ett
nytt kryptografiskt system. Det betyder att vi redan nu måste börja undersöka
och byta till kryptografiska system som är resistenta mot kvantdatorer. I denna
essä har vi presenterat olika alternativ som undersöks för tillfället. Det
finns flera problem vi inte gick in på, som även tros vara kvant-resistenta,
som till exempel att lösa multivariata-polynom.

\end{document}
